\section{Continuous Delivery}
\begin{mdframed}
    \textbf{Continuous Delivery:} approccio di ingegneria del software in cui i team producono software in cicli brevi, assicurando che il software possa essere rilasciato in modo affidabile in qualsiasi momento e, quando lo rilasciano, lo fanno manualmente.
\end{mdframed}
Lo scopo è costruire, testare e rilasciare software con maggiore velocità e frequenza. 
Serve per ridurre i costi, il tempo e il rischio di cambiamenti nel rilascio rendendo maggiormente incrementale il rilascio delle applicazioni in produzione. \\ \\
CD è una disciplina di sviluppo software con la quale il software viene costruito in modo tale che possa essere rilasciato in produzione in qualsiasi momento.
Nella pratica questo significa che:
\begin{itemize}
    \item Il software è rilasciabile attraverso tutto il suo ciclo di vita.
    \item Il team da la priorità a mantenere il software rilasciabile rispetto ad aggiungere nuove funzionalità.
    \item Chiunque può ottenere un feedback automatizzato sulla readiness dei sistemi ogni qualvolta qualcuno fa una modifica.
    \item Si possono ottenere rilasci di qualsiasi versione del software in qualsiasi ambiente a comando.
    \item Si integrano le pratiche di CI del team di sviluppo integrando il building degli eseguibili e eseguendo test automatizzati sugli eseguibili.
\end{itemize}
L’obiettivo è trasformare il rilascio di un sistema di qualsiasi dimensione in un processo prevedibile che può essere eseguito a richiesta.
Il modo per farlo è assicurarsi che il codice sia sempre pronto al rilascio, anche nel caso di cambiamenti continui da parte di team di sviluppatori molto grandi.
Tutte le parti successive alla fine dello sviluppovengono eliminate.

\subsection{Problemi}
La Continuous Integration permette di avere feedback su problemi introdotti
dagli sviluppatori.
Si focalizza sulla parte DEV e assicura che:
\begin{itemize}
    \item Il codice compili
    \item Vengono eseguiti i test di unità, integrazione, accettazione e l’analisi statica
\end{itemize}
Questo non è sufficiente per garantire la possibilità di rilasciare il prodotto ad ogni modifica perché le attività che di solito fanno perdere più tempo avvengono nella fase di rilascio e test (e nella comunicazione e la collaborazione tra DEV TEST e OPS)

\subsubsection{Problemi Comuni}
\begin{itemize}
    \item \textbf{OPS:} i sistemisti aspettano tempo per riceve la documentazione con le procedure di rilascio.
    \item \textbf{TEST:} i tester attendono tempo per effettuare verifiche e validazione nella versione giusta.
    \item \textbf{DEV:} il team di sviluppo riceve segnalazioni di bug su funzionalità che sono state rilasciate da settimane.
    \item \textbf{Archiettura:} ci si rende conto solo alla fine dello sviluppo che l’archittetura scelta non permette di soddisfare i requisiti non funzionali.
\end{itemize}

\subsubsection{Conseguenze}
\begin{itemize}
    \item \textbf{Non rilasciabile:} perchè si è impiegato troppo a farlo entrare in produzione.
    \item \textbf{Contiene difetti:} contiene difetti perchè il ciclo di feedback tra team di development, testing e operations è troppo lungo.
\end{itemize}

\subsection{Deployment Pipeline} 
L’obiettivo è quello di migliorare il processo che permette di rilasciare una modifica al codice sorgente del progetto in produzione.
\begin{multicols}{2}
    \raggedright
    Una \textbf{Value Chain} è una modellazione del processo per misurare:
    \begin{itemize}
        \item Il Valore Globale
        \item Il Valore residuo da inserire nella “Value Chain” (il miglioramento)
        \item[]
    \end{itemize}
    \columnbreak
    Una \textbf{Deployment Pipeline} dà gli stessi risultati di una Value Chain negli altri settori non software:
    \begin{itemize}
        \item Controllare le prestazioni
        \item Migliorarne l'efficienza
        \item "Fast is cheap": le Pipeline permetto di trovare i problemi il prima possibile
    \end{itemize}
\end{multicols}
\begin{mdframed}
    \textbf{Deployment Pipeline}: modellazione del processo di Deployment tramite una successione di fasi (stages) e verifiche (gates).
\end{mdframed}
\begin{itemize}
    \item Il passaggio da una fase all’altra viene verificato tramite il superamento di una verifica
    \item Il passaggio di una fase può scatenare una notifica
    \item È guidato dal concetto di Fail-Fast
    \item È Composta da stages mappate su attività misurabili (build, test)
    \item La transazione tra due stages viene definita gate
    \item Può essere automatica o manuale
    \item I gates scatenano il passaggio allo stage successivo
    \item Gli Stages possono essere eseguiti in parallelo e/o in sequenza
    \item I gates possono essere multi direzionali
\end{itemize}

\subsection{Realizzazione}
\begin{itemize}
    \item Avere un rapporto di lavoro collaborativo con tutti i partecipanti (dev, test e ops)
    \item Utilizzare le Deployment Pipelines per definire il processo di build, test e deploy dell’applicazione.
    \item Distribuire il processo di build e deploy su più ambienti
\end{itemize}

\subsubsection{Requisiti}
\begin{itemize}
    \item \textbf{Continuous Integration:}
    \begin{itemize}
        \item VCS
        \item Build automation
        \item Unit Testing
        \item Artifact Repository
    \end{itemize}
    \item \textbf{Configuration Management:} strumenti che ci permettono, di gestire tramite codice, la configurazione degli ambienti dove dovrà essere rilasciato il nostro software
    \item \textbf{Continuous Testing:} test automatici a livello di sistema funzionali e non funzionali
    \item \textbf{Orchestratore:} sistema software che ci permette di modellare le esecuzioni delle pipeline
\end{itemize}

\subsection{Practices}

\subsubsection{Only Build your Binaries once}
Eseguire il processo di build UNA sola volta. Lo stesso artefatto verrà utilizzato per le verifiche in ogni ambiente.
In questo modo:
\begin{itemize}
    \item L’artefatto che verrà rilasciato in produzione è lo stesso che è stato verificato e validato nelle stages della pipeline
    \item Forma di ottimizzazione (faccio il lavoro solo una volta)
\end{itemize}
È consigliabile avere un Artifact Repository dove rilasciare gli artefatti.
L’artefatto generato deve essere indipendente dall’ambiente di esecuzione.
Tenere il codice separato dall’ambiente di esecuzione.

\subsubsection{Deploy the Same Way to Every Environment}
Utilizzare lo stesso script per effettuare il rilascio in ambienti differenti.
Lo script di rilascio sarà più solido perché verrà verificato maggiormente.

\subsubsection{Smoke-Test your Deployments}
Per verificare se il rilascio è andato bene prevedere l’esecuzione di smoke-test.
Gli smoke-test devono anche verificareil corretto funzionamento dei sub-system esterni
L’obbiettivo è sempre quello del fail fast, bisogna cercaregli errori in modo da far fallire la pipeline il più velocemente possibile.

\subsubsection{Deploy into a Copy of Production}
Prevede di avere a disposizione un ambiente con le stesse caratteristiche (o simili) dell’ambiente di produzione.
Quindi è necessario eseguire i test e gli script di rilascio in ambienti il più possibile simili all’ambiente di produzione.
La gestione dell’ambiente di rilascio è il Configuration Management.
L’ambiente dovrà avere:
\begin{itemize}
    \item La stessa configurazione di rete
    \item Lo stesso SO
    \item Lo stesso stack applicativo
    \item Gestione dei dati consistente
\end{itemize}

\subsubsection{Each Change should Propagate through the Pipeline Instantly}
Ogni modifica al codice sorgente deve avviare la pipeline.
La pipeline impiegà molto tempo per eseguire l’intero processo e servono verifiche negli stages che la facciano fallire il prima possiible.
La CI non eseguirà ad ogni commitè sarà più difficile identificare l’errore.
Per questo si consiglia un server CI che possa eseguire il processo di CI apartire da uno specifico commit.
In questo modo sarà più semplice effettuare attività di debug per identificare la modifica che ha fatto fallire il processo

\subsubsection{If any Part of the Pipeline Fails, Stop the Line}
Eseguire per primi i controlli veloci e meno esaustivi.
In questo modo si può far fallire la pipeline e notificare il team del problema.

\newpage
\subsection{Benefici}
\begin{itemize}
    \item \textbf{Ridurre il rischio legato al deploy:} dal momento che si sta distribuendo piccole modifiche, è meno probabile corrompere il sistema ed è più facile risolvere l’errore in caso di problemi.
    \item \textbf{Velocizza il time to market:} questa pratica porta ad avere rilasci più frequenti
    \item \textbf{Maggiori feedback da parte degli utenti}
    \item \textbf{Progressi tangibili:} molti team diversi monitorano i progressi. Cioè la semplice conferma da parte degli sviluppatori che qualcosa è stato fatto è molto meno forte rispetto a "fatto, verificato e distribuito in un ambiente di
    produzione".
    \item \textbf{Minor costo:} diretta conseguenza dell’automazione
    \item \textbf{Prodotti migliori:} che soddisfano le aspettative degli utenti
    \item \textbf{Team meno stressati e più collaborativi}
    \item \textbf{Maggiore documentazione implicita}
\end{itemize}

\newpage