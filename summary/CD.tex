\section{Continuous Delivery}
\begin{mdframed}
    \textbf{Continuous Delivery:} approccio di ingegneria del software in cui i team producono software in cicli brevi, assicurando che il software possa essere rilasciato in modo affidabile in qualsiasi momento e, quando lo rilasciano, lo fanno manualmente.
\end{mdframed}
Lo scopo è costruire, testare e rilasciare software con maggiore velocità e frequenza. 
Serve per ridurre i costi, il tempo e il rischio di cambiamenti nel rilascio rendendo maggiormente incrementale il rilascio delle applicazioni in produzione. \\ \\
CD è una disciplina di sviluppo software con la quale il software viene costruito in modo tale che possa essere rilasciato in produzione in qualsiasi momento.
Nella pratica questo significa che:
\begin{itemize}
    \item Il software è rilasciabile attraverso tutto il suo ciclo di vita.
    \item Il team da la priorità a mantenere il software rilasciabile rispetto ad aggiungere nuove funzionalità.
    \item Chiunque può ottenere un feedback automatizzato sulla readiness dei sistemi ogni qualvolta qualcuno fa una modifica.
    \item Si possono ottenere rilasci di qualsiasi versione del software in qualsiasi ambiente a comando.
    \item Si integrano le pratiche di CI del team di sviluppo integrando il building degli eseguibili e eseguendo test automatizzati sugli eseguibili.
\end{itemize}
L’obiettivo è trasformare il rilascio di un sistema di qualsiasi dimensione in un processo prevedibile che può essere eseguito a richiesta.
Il modo per farlo è assicurarsi che il codice sia sempre pronto al rilascio, anche nel caso di cambiamenti continui da parte di team di sviluppatori molto grandi.
Tutte le parti successive alla fine dello sviluppovengono eliminate.

\subsection{Problemi}
La Continuous Integration permette di avere feedback su problemi introdotti
dagli sviluppatori.
Si focalizza sulla parte DEV e assicura che:
\begin{itemize}
    \item Il codice compili
    \item Vengono eseguiti i test di unità, integrazione, accettazione e l’analisi statica
\end{itemize}
Questo non è sufficiente per garantire la possibilità di rilasciare il prodotto ad ogni modifica perché le attività che di solito fanno perdere più tempo avvengono nella fase di rilascio e test (e nella comunicazione e la collaborazione tra DEV TEST e OPS)

\subsubsection{Problemi Comuni}
\begin{itemize}
    \item \textbf{OPS:} i sistemisti aspettano tempo per riceve la documentazione con le procedure di rilascio.
    \item \textbf{TEST:} i tester attendono tempo per effettuare verifiche e validazione nella versione giusta.
    \item \textbf{DEV:} il team di sviluppo riceve segnalazioni di bug su funzionalità che sono state rilasciate da settimane.
    \item \textbf{Archiettura:} ci si rende conto solo alla fine dello sviluppo che l’archittetura scelta non permette di soddisfare i requisiti non funzionali.
\end{itemize}


\subsubsection{Conseguenze}
\begin{itemize}
    \item \textbf{Non rilasciabile:} perchè si è impiegato troppo a farlo entrare in produzione.
    \item \textbf{Contiene difetti:} contiene difetti perchè il ciclo di feedback tra team di development, testing e operations è troppo lungo.
\end{itemize}









\newpage