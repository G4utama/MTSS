\section{Analisi Statica}
\begin{mdframed}
    Analisi Statica: analisi di un software che non richiede l'esecuzione del codice per essere eseguita.
\end{mdframed}
A differenza della dinamica che viene eseguita quando il programma è in esecuzione.
Può essere fatta sul codice sorgente o sul codice oggetto.
Il termine è associato all'analisi effettuata da uno strumento automatico, mentre l'analisi fatta da un essere umano è detta, program understanding, program comprehension o code review.
È un tipo di test:
\begin{itemize}
    \item \textbf{Statico:} non ho bisogno di eseguire codice.
    \item \textbf{White box:} ho bisogno di vedere il codice.
    \item \textbf{Non funzionale:} controllo in che modo il è strutturato, non quello che fa.
\end{itemize}

\subsection{Automated Code Review Software}
\begin{mdframed}
    \textbf{Automated code review software:} verifica il codice sorgente per la conformità con un insieme di regole predefinite obest practices.
\end{mdframed}

\subsubsection{Caratteristiche}
\begin{itemize}
    \item Rappresenta una pratica standard.
    \item Può essere fatto sia a mano che automatizzato.
    \item Solitamentemostra una serie di warning con l'elenco delle violazioni riscontrate.
    \item Può anche fornire un sistema per correggere gli errori trovati.
\end{itemize}

\subsubsection{Strumenti}
\begin{itemize}
    \item Possono essere utilizzate per assistere la Automated Code Review.
    \item Non sono efficaci come il controllo umano ma possono essere eseguiti più velocemente. 
    \item Incapsulano al loro interno una conoscenza profonda delle regole e della semantica per eseguire l'analisi
    \item Permettono al verificatore di non necessitare dello stesso livello di esperienza di un esperto del settore.
\end{itemize}

\subsection{Teoria delle Finestre Rotte}
La teoria delle finestre rotte è una teoria criminologica sulla capacità del disordine urbano e del vandalismo di generare criminalità aggiuntiva e comportamenti antisociali.
La teoria afferma che mantenere e controllare ambienti urbani reprimendo i piccoli reati, gli atti vandalici, la deturpazione dei luoghi, il bere in pubblico, la sosta selvaggia o l'evasione nel pagamento di parcheggi, mezzi pubblici o pedaggi, contribuisce a creare un clima di ordine e legalità e riduce il rischio di crimini più gravi. \\ \\
Ad esempio l'esistenza di una finestra rotta potrebbe generare fenomeni di emulazione, portando qualcun altro a rompere un lampione o un idrante, dando così inizio a una spirale di degrado urbano e sociale.

\subsection{Tools}
\begin{itemize}
    \item \textbf{Checkstyle:} strumento di sviluppo che aiuta i programmatori a scrivere codice Java che aderisce a uno standard di codifica. Automatizza il processo di verifica del codice Java.
    \item \textbf{FindBugs/SpotBugs:} programma che utilizza l'analisi statica per cercare i bug nel codice Java.
    \item \textbf{PMD:} analizzatore statico di codice sorgente, che trova i più comuni difetti di programmazione.
    \item \textbf{SonarQube:} strumento di revisione automatica del codice per individuare bug, vulnerabilità e code smell nel codice. \\
    Funzionalità:
    \begin{itemize}
        \item storicizza l'andamento della qualità
        \item verifica se c'è un miglioramento o un deterioramento del progetto nel tempo
        \item stabilisce un insieme di regole da applicare al progetto [quality profile]
        \item verifica se la qualità del progetto rispetta determinati standard [quality gate]
        \item classifica issue in base alla gravità (blocker, critical, major, minor, info)
        \item classifica issue in: vulnerabilità , bug, code smell
        \item revisiona le issue segnalate e segna i falsi positivi
    \end{itemize}
\end{itemize}

\subsubsection{Funzionalità}
Le funzionalità sono simili a quelle di un correttore automatico che segnala un errore. Permettono di:
\begin{itemize}
    \item Imporre il rispetto di convenzioni e stili
    \item Verificare la congruità della documentazione
    \item Controllare metriche e indicatori
    \begin{itemize}
        \item complessità ciclomatica
        \item grafo dipendenze
        \item numerosità righe di codice
    \end{itemize}
    \item Ricercare codice copiato in più punti
    \item Ricercare errori comuni nel codice
    \item Misurare la percentuale di codice testato
    \item Ricercare indicatori di parti incomplete
\end{itemize}

\newpage